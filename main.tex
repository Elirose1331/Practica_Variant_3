%% -*- coding: utf-8 -*-
\documentclass[12pt,a4paper]{scrartcl} 
\usepackage[utf8]{inputenc}
\usepackage[english,russian]{babel}
\usepackage{indentfirst}
\usepackage{misccorr}
\usepackage{graphicx}
\usepackage{amsmath}
\begin{document}
\section{Введение}
\label{sec:intro}

% Что должно быть во введении
\begin{enumerate}
    \item Текстовая формулировка задачи
    \item Код приложения
    \item Пример формулы
    \item Скриншот программы
    \item Пример библиографических ссылок
\end{enumerate}

\section{Текстовая формулировка задачи}
\label{sec:exp:code}
\begin{mat}
В данном приложении мы представляем алгоритм реализации метода секущих для нахождения корня функции.
Примечание:
В методе Ньютона требуется вычислять производную функции, что не всегда удобно. Можно заменить производную первой разделённой разностью, найденной по двум последним итерациям, т. е. заменить касательную секущей. Это и называется методом секущих. 

\end{mat}

\section{Ход работы}
\label{sec:exp}

\subsection{Код приложения}
\label{sec:exp:code}
\begin{verbatim}
def F( x : float):
  return 2*(x**3)-x**2 - 0.46

def search(f: float , a : float , b : float ,  eps : float):
  while(abs(a - b) > eps):
         a = b - (b - a) * f(b) / (f(b) - f(a))
         b = a - (a - b) * f(a) / (f(a) - f(b))
  return b
  
a = float(input('Введите границу а : '))

b = float(input('Введите границу b : '))
 
eps = float(input('Введите погрешность : '))

x = search(F , a , b , eps)
print("Корень : " , x)

\end{verbatim}

\subsection{Пример формулы}
\label{sec:exp:code}
\begin{mat}
 x_{n+1} = x_{n-1} - \dfrac{f(x_{n})\cdot(x_{n} - x_{n-1})}{f(x_{n}) - f(x_{n-1})}
\end{mat}
\section{Скриншот программы}
\label{sec:picexample}
\begin{figure}[h]
	\centering
	\includegraphics[width=1.0\textwidth]{codeLiza.PNG}
\end{figure}


\section{Пример библиографических ссылок}

Для изучения «внутренностей» \TeX{} необходимо 
изучить~\cite{Knuth-2003}, а для использования \LaTeX{} лучше
почитать~\cite{Lvovsky-2003, Voroncov-2005}.

\begin{thebibliography}{9}
\bibitem{Knuth-2003}Кнут Д.Э. Всё про \TeX. \newblock --- Москва: Изд. Вильямс, 2003 г. 550~с.
\bibitem{Lvovsky-2003}Львовский С.М. Набор и верстка в системе \LaTeX{}. \newblock --- 3-е издание, исправленное и дополненное, 2003 г.
\bibitem{Voroncov-2005}Воронцов К.В. \LaTeX{} в примерах. 2005 г.
\end{thebibliography}

\end{document}
